%\documentclass[prl,twocolumn,showpacs]{revtex4}
\documentclass[prl,showpacs]{revtex4}
%%%%%%%%%%%%%%%%%%%%%%%%%%%%%%%%%%
\usepackage[dvips]{graphicx}
\usepackage{latexsym,amsmath,amssymb,amsthm,xspace,times}
\usepackage{color}
%%%%%%%%%%%%%%%%%%%%%%%%%%%%%%%%%%%%%%%%%%%%%%%%%%%%%%%%%%%%%%%%%%%%%%%%%%%%%%%%%%%%%
\begin{document}
\title{Effects of pleiotropy on the specificity and accuracy of receptor signaling.}

\author{A. Zilman$^{1,2}$}
\affiliation{
$^{1}$Department of Physics, Faculty of Arts and Sciences,University of Toronto, Toronto, ON M5S 1A7, Canada \\
$^{2}$Institute for Biomaterials and Bio-engineering, Faculty of Engineering\\
University of Toronto, Toronto, ON M5S 1A7, Canada }
% Revision date - uncomment to exclude date in the final version

\begin{abstract}
Functioning of higher organisms requires that cells perform different functions at different times. Collective function. Infirmation exchange. Solube signaling molecules bind to the cell surface receptors and induce sugnaling cascades culminating in differentn action, such as movement, differentiation or secretion of other signaling and effector molecules. These molecules often appear in multiple combinations and cells need to respond differently to different signals and also to different concentrations. Commonly pathways shoe significant cross talk, and its shadow sister, pleiotropy. This happens on the receptor level to some extent, adn to a very large extent on the level of the downstream regulator. Sinaling specificty refers to how well a cell can distinguish between different ligands of different affinities, while the specificity refers to the ability of a cell to accurately gauge the concentration of a particular ligand. In cross-wired pleiotropic pathways, it still remains unclear how a cell can accurately determine both, as the molecualr constraints on formation of the ligand bound signaling complex placw severe limitations on the simulataneously both accurate and specific signaling because different affinities can be masked by different concentrations. However, simultanelus sencing speciically and acuratelyis a fundamentally difficult oproblem, because the receptor occupancy reflects both. A number of solutions have been proposed how cell can have asolute specificity an discrimination unconfounded by concentration effects. Some other work examined the limits on how two different ligand concentrations may be estimeted by one receptor on the receptor level. The general pictiure remians missing especially for downstream meessneger molecularly realistic. In this paper we frqmce the problem of specificity as the estimation of the ligand affinity to the receptor. We show that the specificty-accuracy tradeoff can be resolved by pleiotropic signaling and caluclate the expected exstinmaTE accuracies. We also look at KPR secheme and see how it improves things.
\end{abstract}
\pacs{87.10.Ca, 87.10.Mn, 87.85.Rs}
\maketitle
\emph{Background}:  In order to function in dynamic and complex environments, cells have evolved molecular mechanisms to to sense and respond to their environment and communicate with other cells.  A major mode of sensing and inter-cellular communication is through signaling receptors that bind signaling molecules to convey information about the environment either directly from the environment (such as food, repellants or attractants) or from other cells. Receptor signaling is the main mode of multi-cellular coordination during development, in the nervous and immune systems, and other physiological circuitry in multi-cellular organisms. Typically, binding of a signaling molecule to a cell surface receptor activates a cascade of intra-cellular events eventually leading to action in the form of changes in motility (eg. bacteria, chemokines), phenotypical changes (eg. T cell or macrophage activation) or cellular  differentiation (eg. haematopoiesis, development).

Signaling molecules convey different types of information about the enviromnemt. Naturally, different signals elicit different responses, and the response may also be dose dependent. Depending on the circumstances, specific signaling pathways have evloved to optimize transmission of different features of the signal information. For instance, in some cases the function of the signaling pathway is to respond to very small levels of the signaling molecules - thus focusing on sensitivity. In other cases, specificity - that is the ability to distiguish between molecules with similar chemical structure - might be of importance (eg. T cell receptors). Or, the cell might be trying to rapidly and accurately estimate small differences in the concentration, such as is the case in bacterial chemotaxis - a classical paradigmal problem. Many of these functions involve trade-offs [Mehta, others]: for instance, maximizing the sensitivity through very strong ligand binding diminishes the accuracy due to receptor saturation []. Specificity and accuracy are also often at odds, as discussed in more detail below.  [KPR literature?]. These tasks are performed in the presence of molecular and environmental noise that masks the signal.

The questions of accuracy and specificity have been investigated extensively in the past decade in the theoretical literature, and are starting to impact experimental work [Adaptive sorting, bacterial literature?, neurons?]. The question of accuracy, dating back to the seminal work of Berg and Purcell [] asks what the limits are on the accuracy of concentration sensing from receptor occupancy in the presensce of molecular noise. In particular, it has been shown that if the cell has access to the whole time series of binding and unbinding events, it estimates concentration with maximal accuracy. In practice, it boils down to knowing both the bound time of the receptor and the number of binding and unbinding events. This model only considers one type of receptor. It is not clear how the cell might have access to the whole binding/unbinding sequence or how molecular noise in the downstream signaling cascade might impact this estimate. On the other hand, other work has focused on mechansims to increase specificity beyond the Boltzman factor. A typical measure of the specificty is the difference in the signal between two different ligands at the same concentration []. The canonical mechanism for increasing specificity is the kinetic proofreading, which increases specificty at the expense of longer time spent estimating and non-equilibrium energy input [lots of KPR refs]. There have been modifications of KPR and applications to experiments [GAB, Goldstein/Faeder, Other mechanisms- Dushek].

An important and very common property of many signaling pathways is cross-talk, whereby one ligand binds multiple receptors to activate multiple downstream pathways []. Although some systems are remarkably one-to-one (bacteria, some cytokines), the problem of cross-talk has been puzzling because it raises the question of how the system can determine accurately and specifically the inputs from the mixture of activated downstream pathways. In principle this is not a fundamental problem, and the same question arises in signal processing such as blind source decomposition, and has been resolved in that literature. However, solving this problem with only the tools available to a cell is a more difficult problem. The effect of cross-talk has been recently attempted to be addressed in several works following Berg-Purcell-Wingreen formalism to calculate how concentrations of ligands could be accurately sensed in mixtures of two ligands by either single or two receptors [Mora, Nemenman]. This works for a mixture of \emph{two} known signaling molecules, but it seems difficult to generalize to many signals mapping through many receptors and still gain meaningful insight. Notably, cross-talk is closely related to pleiotropy, whereby one receptor can activate multiple downstream pathways, and in fact this is an integral part of cross talk. There are also questions of how transcriptional regulation helps cells dientangle these signals (Tkacik).

In reality, the main problem is that the cell often sees unknown combinations of multiple signaling molecules that bind to the recetopr with \textit{a priori} unknown rates. In practice, this means that the knowledge of the binding and unbinding of different ligands to a receptor is not molecularly encoded into the molecular inference network. To illustrate this, consider the simple example of a receptor that can be bound by multiple different molecules. The measurement of its receptor occupancy, or any downstream function thereof, is not sufficient for knowledge of both the ligand identity \emph{and} its concentration, because the same occupancy statistics can be produced by either a higher concentration of lower affinity ligand, or the lower concentration of the higher affinity ligand, as illustrated in Fig...  Mathematically, the concentration and the affinity enter in a combination $\exp(\mu-\epsilon)/k_BT=c/K$ into every analysis.

Some suggestions for absolute discrimination of ligands unconfounded by the signal concentration have been made [us, Paul] but the problem remains open. More generally, when a cell sees a combination of $n$ unknown cytokines it is impossible to ascertain both what combination it is and what are the concentrations of its components, even if there are $n$ receptors, so no funnel.[QUESTION: if we have twice more receptors than liagnds - are we ok? - I dont think so.] The answer to this problem may have important applications for molecular sensing, whereby a sensor needs to estimate concentrations of combinations of closely related but unknown chemicals with potential cross-reactivity (eg. antibody ELISA, nanochannels).

In this paper, we formulate this problem as the need to sense both the ligand concentration and the affinity of the ligand to the receptor. Beacuse it is mathematically impossible to unambigously determine two variables from one measured variable, the inference scheme has to have two output variables. There are a number of possible scenarios, but we choose to focus on the case where the system can measure both the bound time on the receptor and the number of binding/unbinding events, to benchmark against the optimal result of Wingreen and Endres, in a molecularly realistic scheme.

To measure the bound time is easy, and that is what most receptors do. This measurement mode is shown in Figure 1, where the bound receptor produces a signal, for instance via phosphorylation of substrate at rate $k_p$. The other mode is shown in Figure... and produces one signaling molecule every time the ligand unbinds (or binds) to the receptor. One molecular realization of such signaling is GPCR-type receptors, where binding of the ligand causes the unbinding of the $G_{\alpha}$ subunit protein from the receptor, which subsequently acts as a downstream signal. Other realizations can include rapid receptor internalization or entry into another refractory state on a timescale faster than the ligand unbinding, whereby each binding event produces a batch of signaling molecules whose size is independent of the unbinding rate of the ligand. We consider two paradigmatic signaling types. One, common to a large number of pathways (cytokine and growth factor signaling, TCR, chemotaxis), where the receptor produces a signal (typically in the form of phosphorylated signaling molecules) while the receptor is bound. The variable that is typically considered in this case is the receptor occupancy, or the bound/unbound time. Once the ligand uinbinds, the singaling is interrupted. In the other large class of receptors - exemplified by GPCRs - the binding of the ligand causes dissociation of the signaling complex from the receptor, which subsequently produces signal until it is degraded. In this case, each binding event produces a burst of signal which is not directly linked to the binding time of the receptor but rather reflects the number of binding/unbinding events. We consider a toy model where one ligand binds to two receptors, one of each class. This allows us to estimate the $(K_d, c)$ from the outputs of the signaling $(n, m)$ provided that they contain different information.

\emph{The model}.


\begin{figure}
    \includegraphics[width=0.4\textwidth]{fig-sensing-horizontal.eps}
    \caption{Schematic illustration of two modes of signaling. First is the typical signaling , where the signaling molecule is prodiuced while the ligand is bound. The second is GPCR-like, and the signaling molecule is released from the receptor upon ligand binding.}
\end{figure}

We first discuss the signaling in Mode 1, to put it in the context of previous work. In the model, the receptor is in contact with the ligand solution present at concentration $c$. The ligand binding rate is $k_{\text{on}}$ and the unbinding rate is $k_{\text{off}}$. In the bound state, the receptor produces signal at rate $k_p$. Mathematically, the receptor can be in two states, bound and unbound, and transitions between them with rates $k_{on}c$, $k_{\text{off}}$, corresponding to ligand binding and unbinding, respectively. These rates are related to the energy of binding through the equilibrium dissociation constant $K_d=k_{\text{off}}/k_{on}\simeq e^{-\epsilon}$. At steady state, the probability occupancies are $\vec{P}_{ss}=(1/(1+x),x/(1+x)$, where $x=c/K_d$.

The Master equation that describes the probability of the system to be in state 1 or 0 and $n$ signal molecules produced, $P_{0,1}^n(t)$ is [nemenman-sinitsyn, Iyer-Biswas]:
%
\begin{eqnarray}
\frac{d}{dt}P_0^n(t)&=&k_{\text{off}}P_1^n(t)-k_{on}c P_0^n(t)\\
\frac{d}{dt}P_1^n(t)&=&k_{on}cP_0^n(t)-k_{\text{off}} P_1^n(t)+k_pP_1^{n-1}(t)- k_p P_1^n(t)
\end{eqnarray}
%
Defining the generating function $G_i(s,t)=\sum_n s^n P_i^n(t)$, we get for the vector $\vec{G}=(G_0(s,t),G_1(s,t)$.
%
\begin{eqnarray}
\frac{d}{dt}\vec{G}(s,t)=\hat{M}\cdot\vec{G}(s,t)\;\;\text{ with}\; \hat{M}=\begin{bmatrix}-k_{on}c &k_{\text{off}}\\k_{on}& -k_{\text{off}}+k_p(s-1)\end{bmatrix}
\end{eqnarray}

The generating function is then obtained as $\vec{G}(t)=e^{\hat{M}t}\cdot \vec{G}(0)$. Assuming that the receptor is at steady state at the beginning of the measurement, the initial condition is $\vec{P}^n=\vec{P}_{ss}\delta_n,0$ and therefore $\vec{G}(0)=\vec{P}_{ss}$. Similar results can be obtained assuming that the receptor is initially unoccupied (see Supplementary Material) They produce identical results in the limit of large numbers of events, $k_{\text{off}}t\gg 1$ which is the focus of this paper; In the Discussion we touch upon the the results in the short time regime. The mean and the variance of the distribution of $n$ are readily calculated as $\langle n\rangle=\frac{\partial G(s,t)}{\partial s}|_{s=1}$ and
$var(n)=\frac{\partial^ G(s,t)}{\partial s^2}|_{s=1}+\langle n\rangle-\langle n\rangle^2$:
%
\begin{eqnarray}
\langle n\rangle&=&k_pt\frac{x}{1+x}\\
Var(n)&=& k_p t\frac{x}{1+x}+\\
&+& \frac{2k_{p}^2t}{k_{\text{off}}}\frac{x}{(1+x)^3} \left(1+\frac{e^{-k_{\text{off}} t (x+1)}-1}{k_{\text{off}}t(x+1)}\right)\nonumber.
\end{eqnarray}
%
At short times, $n$ obeys a Poisson distribution with the mean $\langle n\rangle$, while for large $k_{\text{off}}t\gg 1$, it tends to  the Normal distribution with the mean  $\langle n\rangle$ and the variance that tends to $Var(n)=k_ptx/(1+x)+2k_{p}^2 t x/(k_{\text{off}}(1+x)^3)$, following from Central Limit Theorem; we confirmed in numerical simulations that for $k_{\text{off}}\approx 10$, it is already well described by the Normal distribution (not shown). See also the Supplementary for heuristic derivation of these results. These results are intuitive: on average, the receptor produces molecules with the rate $k_p$ all the time it is occupied, ie. on average $x/(1+x)$ fraction of the time $t$.

This simple model illustrates the accuracy-specificity trade-off. Given $k_{\text{off}}$ and $k_{\text{on}}$ - which is effectively equivalent to the assumption that the molecular specificity of the ligand-receptor binding is very high, and the probability of the receptor to bind any other ligand except its cognate one is very low - it is possible to estimate $x$ (and therefore the concentration $c$) from the measurement of the number of signaling molecules $n$.  A good heuristic estimate is $x^*=n/k_pt/(1-n/k_pt)$, and the variance of this estimate is $\sigma^2_x\simeq Var(n)(\frac{dx^*}{dn})^2$ provided that the distribution $P(x|n)$ is reasonably peaked around $x^*$.

This heuristic estimate agrees with the more rigorous maximum likelihood Bayesean approcach. Mathematically, we need to estimate $x$ based on $n$ and we are interested in $P(x|c)=P(c|x)P(x)/\int dx P(c|x)$. For uniform prior $P(x)$, maximization of this probability over $x$ reduces to the ML estimation based on maximization of the conditional probability. Thus, the relative accuracy of the esitmate of the concentration (details in the Supplementary)
%
\begin{equation}
\frac{\sigma^2_x}{x^2}=\frac{1}{k_p t}\frac{1+x}{x}((1+x)^2+2\frac{k_p}{k_{\text{off}}})=\frac{1}{\langle n\rangle}((1+x)^2+2\frac{k_p}{k_{\text{off}}})
\end{equation}\label{delta-x-from-n}
%
assuming that the the observation results in a typical measurement, where $n\simeq k_pt x/(1+x)$ and $x\simeq x^*$.

For fixed $k_{\text{off}}$, this translates into the estimate of the accuracy of the concentration sensing as $\sigma^2_c/c^2=\sigma^2_x/x^22$ using $c=K_d x$. Altenernatively, for the specificty question, the binding affinity estimate is $K_d^*=x^*/c$ and its variance is $(\delta K_d)^2=(-c/x^2)^2(\delta c)^2=(\delta c)^2/c^2=(\delta x)^2/x^2$. Note that this resolution estimate is different from the common measure of the specificty as the ratio of the of the ligand affinites. In the limit $k_p\gg k_{\text{off}}$ the expression of reduces to the classical Berg-Purcell expression, as in this limit each binding event produces multiple signaling molecules, and the sensing accuracy is limited by the fluctuations in receptor occupancy (see also Supplementary material).

As expected, the estimated accuracy of $c$ or $K_d$ diverges both for $x\rightarrow 0$ and $x\rightarrow \infty$ because at very low concentrations the receptor does not produce enough signaling molecules for a meaningful statistical estimate, while at very high concentrations the receptor occupancy saturates to a constant that is independent of either affinity or concentration, precluding accurate determination of either.

\iffalse
\begin{figure}
    \includegraphics[width=0.45\textwidth]{fig-discrimination-problem-dose-response.eps}
     \includegraphics[width=0.45\textwidth]{fig-x-relative-accuracy.eps}
    \caption{Left panel: Signal discrimination problem. Each curve shows $n$ for a particular $K_d$. Dashed lines are the standard deviations of the signal. For a given concentration, their affinities can barey be distinguished. For fixed concentrations (or in the saturation limit) they also can be barely distinguished. The main problem is that any signal could be produced either by a lower concentration of higher affinity ligand, or a higher concentration of low affinity ligand. Right panel: relative accuracy of determination of x from n.}
\end{figure}
\fi

\begin{figure}
    \includegraphics[width=1.\textwidth]{fig-discrimination-combine-x-relative.eps}
    \caption{Left panel: Signal discrimination problem. Each curve shows $n$ for a particular $K_d$. Dashed lines are the standard deviations of the signal. For a given concentration, their affinities can barey be distinguished. For fixed concentrations (or in the saturation limit) they also can be barely distinguished. The main problem is that any signal could be produced either by a lower concentration of higher affinity ligand, or a higher concentration of low affinity ligand. Right panel: relative accuracy of determination of x from n.}
\end{figure}

%
\emph{Pleiotropy}. Given just the measurement of one variable, $n$, it is impossible to determine \emph{both} $c$ and $k_{\text{off}}$, which means that the system cannot distinguish between a lower concentration of its cognate ligand and a lower concentration of a non-specific one, and it does not know which ligand is bound, as shown in Figure ... . However, if the ligand produces an additional signal - pleiotropy - it is possibe to unambiguosly determine $k_{\text{off}}$ (or $K_d$) \emph{and} $c$ simultaneously.

In this section we consider the second signal to be GPCR-like, so that a signaling molecule $m$ detaches from the receptor every time the ligand binds and then goes on to produce downstream signal, effectively serving as a device for counting the number of binding events. For simplicity, we assume that the rebinding of this G-protien-like molecule occurs quickly. Other molecular realizations could include some receptors being targeted for internalization while others are not. 

The system is now described by the two variable probability distribution $P_i^{n,m}$  which denotes the probability of the system to contain $n$ n-like molecules and $m$ m-like molecules at time $t$, and for the receptor to be in state $i$. It is described by the following Master equation [Nemenman-sinitsyn]:

\begin{eqnarray}
\frac{d}{dt}P_0^{n,m}(t)&=&k_{\text{off}}P_1^{n,m}(t)-k_{on}c P_0^{n,m}(t)\\
\frac{d}{dt}P_1^{n,m}(t)&=&k_{on}cP_0^{n,m-1}(t)-k_{\text{off}} P_1^{n,m}(t)+k_pP_1^{n-1,m}(t)- k_p P_1^{n,m}(t)
\end{eqnarray}

This Master equation can be solved using the generating function technique, similar to Equation ... (see details in the Supplementary Material). The mean and the variance of $n$ stay the same as in Eq. (). The mean of $m$, its variance $\text{var}(m)$ and the covariance $\text{cov}(n,m)$ are, in the $k_{off}t\gg 1$ limit, 
%
\begin{eqnarray}
\langle m\rangle&=&k_{\text{off}}t\frac{x}{1+x}\\
\text{var}(m)&=& k_{\text{off}}tx\frac{1+x^2}{(1+x)^3}\nonumber\\
\text{cov}(n,m)&=&k_pt x\frac{1-x}{(1+x)^3}\nonumber
\end{eqnarray}
%
These results can also  be understood based on renewal process theory, as shown in the Supplementary Material. Note that that at small $x$ , $n$ and $m$ are correlated because, at low density/weak binding limit, the binding events are rare and the overall bound time is proportional to the number of events. By contrast at large $x$, $n$ and $m$ are anti-correlated because in this regime the recetopr is occupied most of the time and the time series with more binding-unbinding events results in less bound time. But at very large $x$, $\text{cov}\rightarrow 0$ because the receptor is occupied all the time, and the number of events is not correlated with the total binding time.

\emph{Estimation of $c$ and $K_d$ from $n$ and $m$.}
The two variables $n$ and $m$ allow us to estimate both the concentration $c$ and the affinity $K_d$. In the estimate, we will assume that $k_p$ and $k_{on}$ are constants, hardwired into the cellular machinery. More generally, one can assume that both $k_{\text{off}}$ and $k_{\text{on}}$ depend on the binding energy and estimate the latter. For $k_{\text{off}}t\gg 1$, the probability distribution $P(n,m|c,K_d)\equiv P_1^{n,m}+P_1^{n,m}$ is well approximated by the Gaussian distribution
$$
P(n,m|c,K_d)\simeq \exp(\begin{pmatrix} n-\langle n\rangle(x) \\ m-\langle m\rangle(x) \end{pmatrix} \Sigma^{-1} \begin{pmatrix} n-\langle n\rangle(x) \\ m-\langle m\rangle(x) \end{pmatrix}^{T})
$$
where the covariance matrix is
\begin{eqnarray}
\Sigma^{-1} = \begin{bmatrix}1/\sigma^2_n &1/\sigma_{nm}\\1/\sigma_{nm}& 1/\sigma^2_m\end{bmatrix}=\begin{bmatrix}\text{var}(n) &\text{cov}(n,m)\\\text{cov}(n,m)& \text{var}(m)\end{bmatrix}^{-1}.
\end{eqnarray}
The best estimates for $c$ and $K_d$ can be found my maximizing the likelihood $L(c,K_d)\equiv P(n,m|c,K_d)$ over $c$ and $K_d$, which yields for the best estimates:
%
\begin{equation}
c*=K_d^*\frac{n/(k_pt)}{1-n/k_pt}\;\;\;\; K_d^*=\frac{k_p}{k_{on}}\frac{m}{n}
\end{equation}
%
Note that these estimates can be rewritten as [] and are identical to the theoretically best possible ones obtained from the full likelihood of the whole binding-unbinding series of Eq. ().
The accuracy of these estimates is given by the Fisher Information Matrix [] (see Supplementary), whose elements are
\begin{equation}
\hat{F}=\begin{bmatrix}
    \langle \frac{\partial ^2\ln L(n,m|c,K_d)}{\partial c^2}\rangle & \langle \frac{\partial ^2\ln L(n,m|c,K_d)}{\partial c\partial K_d}\rangle\\
    \langle \frac{\partial ^2\ln L(n,m|c,K_d)}{\partial c\partial K_d}\rangle & \langle \frac{\partial ^2\ln L(n,m|c,K_d)}{\partial K_d^2}\rangle
    \end{bmatrix}
\end{equation}
where averaging is over the distribution of possible values of $n$ and $m$ at a given $c$ and $K_d$. In the context of receptor signaling, it is common to approximate the Fisher Information Matrix by its value around the peak of the distribution $n*=\langle n\rangle$, $m^*=\langle m\rangle$, using a saddle point aproximation. This yields for the estimate accuracies:

\begin{eqnarray}
(\delta c)^2/c^2 & = \frac{(1+x)(x-1)(1+x^2)(2k_p + k_{\text{off}}(1+x)^2)}{k_{\text{off}} t x (k_{\text{off}} (1+x)^2 (3-x+2x^2)-k_p(3x-5x^2+x^3-7)} \\
\delta^2K_d/K_d^2 & = \frac{k_{\text{off}}(x-1)x(1+x)(1+x^2)(2k_p+k_{\text{off}}(1+x)^2)}{c^2k_{\text{on}}^2 t (k_{\text{off}}(1+x)^2(3-x+2x^2)-k_p(3x-5x^2+x^3-7)}
\end{eqnarray}

\begin{figure}
    \includegraphics[width=1.0\textwidth]{fig-delta-c-delta-K-m-n.eps}
    \caption{Accuracy and specificity of pleiotropic signaling. Black: $\langle \delta c^2\rangle /c^2$. Red: $\langle \delta K_d^2\rangle/K_d^2$.}
\end{figure}

\emph{Kinetic proofreading}.
%
\begin{figure}
    \includegraphics[width=0.45\textwidth]{fig-KPR-sensing-horizontal.eps}
    \caption{Schematic illustration of two modes of signaling with Kinetic Proofreading. The KPR mechanism adds an additional "proofreading" state and is often invoked as an explanation for increased specificity.}
\end{figure}
%
A classical scheme to improve signaling specificity, which has been discussed in many contexts, is Kinetic Proofreading (KPR), which enhances the diffferences between ligands of different affinities by introducing additional "proofreading" steps from the binding to the signal production (at the expense of slower sensing time). The scheme is depicted in Fig.... The corresponding master equation in this case is
%
\begin{eqnarray}
\frac{d}{dt}P_0^n(t)&=&k_{\text{off}}P_1^{n,m}(t)-k_{on}c P_0^{n,m}(t)\nonumber\\
\frac{d}{dt}P_1^n(t)&=&k_{on}cP_1^{n,m}(t)-k_{\text{off}} P_1^{n,m}(t)-k_fP_1^{n,m}(t)\\
\frac{d}{dt}P_2^n(t)&=&k_{f}P_1^{n,m-1}(t)-k_{\text{off}} P_2^{n,m}(t)+k_pP_2^{n-1,m}(t)- k_pP_2^{n,m}(t)\nonumber
\end{eqnarray}

Once again using the generating function method, we get
%
\begin{eqnarray}
\langle n\rangle &=&kt\frac{k_f}{k_f+k_{\text{off}}}\frac{x}{1+x}\\
\langle m\rangle &=&k_{\text{off}}t\frac{k_f}{k_f+k_{\text{off}}}\frac{x}{1+x}
\end{eqnarray}
%
As expected, the average number of signaling molecules differs from the non-proofread case by the factor $\frac{k_f}{k_f+k_{\text{off}}}\equiv 1/(1+g)$, where $g=k_{\text{off}}/k_f$. The fluctuations in the numbers of $n$ and $m$ and their covariance are
%
\begin{eqnarray}
\text{var}(n)&=&\frac{2 k_p^2 t x \left(g^2 (x+1)^2+g (x+2)+1\right)}{k_{\text{off}}(g+1)^3 (x+1)^3}+\frac{k_p tx}{(g+1) (x+1)}\nonumber\\
\text{var}(m)&=&\frac{k_{\text{off}} t x \left(1+x^2+g(3+x^2)+g^2 (3+2x+x^2)+g^3(1+x)^3\right)}{(g+1)^4 (x+1)^3}\\
\text{cov}(n,m)&=& \frac{k_{\text{off}} t x \left(1-x+g(1-x)(2+x)+g^2 (x+1)^2\right)}{(g+1)^3 (x+1)^3}\nonumber
\end{eqnarray}

In the limit $k_f\gg k_{\text{off}}$ these expressions reduce to the non-KPR expressions of Equations (), as expected. In the strong proofreading regime, $g\gg 1$ [], Equations () simplify to
%
\begin{eqnarray}
\text{var}(n)=k_pt\frac{1}{g}\frac{x}{1+x}(2\frac{k_p}{k_{\text{off}}}+1)\\
\text{var}(m)=k_{\text{off}}t\frac{1}{g}\frac{x}{1+x}
\end{eqnarray}
%
The accuracy of estimates of $c$ and $K_d$ are calculated in the same fashion and in the strong proofreading regime $g\gg 1$ are (See Supplementary for details):
%
\begin{eqnarray}
\frac{\langle\delta c^2\rangle}{c^2}=g\frac{1}{k_{\text{off}}t}\frac{1+x}{x}(1+2x^2+\frac{k_{\text{off}}}{k_p}x(2+x)\\
\frac{\langle\delta K_d^2\rangle}{K_d^2}=g\frac{1}{k_{\text{off}}t}\frac{1+x}{x}(3-\frac{k_{\text{off}}}{k_p})
\end{eqnarray}
%
[POSSIBLY ONLY SHOW EXPRESSIONS FOR $k_{off}/k_p\gg 1$]
As follows from these equations, KPR does not necessarily enhance the accuracy or specificity of signaling.  Although surprising at first glance, the reason for this was already noted in the original paper by McKeithan, and can be traced to the expressions for the mean and the variance of the signaling molecules in Equations (). Although KPR increases the ratio of the average number of molecules produced by two different ligands by a factor $g$ compared to the non-proofread case, it does so at the expense of a decreased number of signaling molecules, also by a factor of $g$. However, the standard deviation of the product decreases only as $g^{1/2}$, and the relative error increases with KPR. Intuitively speaking, if the level of the non-proofread signal is $\bar{n}\pm\delta n$, the level of proofread signal is $\frac{1}{g}\bar{n}\pm\frac{1}{g^{1/2}}\delta n$. So switching on kinetic proofreading can make non-overlapping clouds overlap, as shown in Fig...
This can potentially be rectified by stabilizing the receptor in the final state of the KPR cascade \cite{McKeithan-KPR-1995} or adaptive feedback schemes %\cite{Francois-Altan-Vergassola-PNAS-2013}\cite{Francois-Altan-JSTAT-2015}\cite{Lipniacki-Faeder-JTB-...}
 and lies outside of the scope of the present work.

\emph{Mixture of ligands}. So far we have considered a case where the ligand is presented alone. In many cases, it is of interest to know what happens in a mixture of ligands. The questions asked can be different: in some cases, the system needs to sense the actual concentrations of both ligands, and sometimes only to threhold the signaling on a particular affinity, or fliter out the non-specific spurious ligands. The general problem of determining both the affinities and the concentrations of both ligands requires more outputs. Such additional outputs could be, for instance, the threshold crossing times for m and n. In other cases, when the task is to determine the concentration of one ligand  from a sea of spurious ones, this might be easier. These cross-talk questions require more detailed modeling of early kinetics and will be studied elsewhere.

\end{document}

